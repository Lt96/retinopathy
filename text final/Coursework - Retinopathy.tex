\documentclass[12pt,fleqn]{article}

\usepackage[utf8]{inputenc}
\usepackage[T1,T2A]{fontenc}
\usepackage{amssymb,amsmath,mathrsfs,amsthm}
\usepackage[unicode, colorlinks, linkcolor=black]{hyperref}
\usepackage[russian]{babel}
%\usepackage[caption=false]{subfig}
\usepackage{graphicx}
\usepackage{caption}
\usepackage{subcaption}
%\usepackage[footnotesize]{caption2}
\usepackage{indentfirst}
\usepackage[backend=biber, sorting=none]{biblatex}
%\usepackage[sorting=none,isbn=false]{biblatex}
%\usepackage[ruled,section]{algorithm}
%usepackage[noend]{algorithmic}
%\usepackage[all]{xy}
%\usepackage{makeidx}
\usepackage[noend]{algpseudocode}
\usepackage{algorithm}

% Параметры страницы
\textheight=24cm
\textwidth=16cm
\oddsidemargin=5mm
\evensidemargin=-5mm
\marginparwidth=36pt
\topmargin=-1cm
\footnotesep=3ex
%\flushbottom
\raggedbottom
\tolerance 3000
% подавить эффект "висячих стpок"
\clubpenalty=10000
\widowpenalty=10000
\renewcommand{\baselinestretch}{1.1}
\renewcommand{\baselinestretch}{1.5} %для печати с большим интервалом

\addbibresource{references.bib}
\bibliography{references.bib}
\DeclareFieldFormat{title}{\mkbibquote{#1}}

%\bibliography{references}{}
%\bibliographystyle{plain}

\begin{document}

\begin{titlepage}
\begin{center}
Московский государственный университет имени М. В. Ломоносова

\bigskip
\includegraphics[width=50mm]{pics/msu.png}

\bigskip
Факультет Вычислительной Математики и Кибернетики\\
Кафедра Математических Методов Прогнозирования\\[10mm]

\textsf{\large\bfseries
КУРСОВАЯ РАБОТА СТУДЕНТА 317 ГРУППЫ\\[10mm]
<<Построение признаков снимков глазного дна для анализа болезней глаза>>
}\\[10mm]

\begin{flushright}
\parbox{0.5\textwidth}{
Выполнил:\\
студент 3 курса 317 группы\\
\emph{Севастопольский Артем Михайлович}\\[5mm]
Научный руководитель:\\
д.ф-м.н., профессор\\
\emph{Дьяконов Александр Геннадьевич}
}
\end{flushright}


\vspace{\fill}
Москва, 2016
\end{center}
\end{titlepage}

\newpage
\renewcommand{\contentsname}{Содержание}
\tableofcontents

\newpage
\section{Введение}

Задачи анализа глазных снимков и компьютерной диагностики глазных болезней представляют собой интересные и важные задачи компьютерного зрения. Компьютерная диагностика глазных болезней позволяет уточнить диагноз, сделанный врачом, за счет улучшения качества изображения и использования архива известных случаев. Некоторые глазные болезни тяжело диагностировать при отсутствии ярко выраженных симптомов или при низком качестве снятого изображения. Диагностика болезней глаз на ранней стадии позволяет назначить большему числу пациентов раннее лечение и тем самым предотвратить развитие болезни. К распространенным болезням, которые могут быть диагностированы с помощью анализа снимков глазного дна, относятся: глаукома, диабетическая ретинопатия (глазной диабет), возрастная дегенерация макулы и др. Данные болезни являются одними из самых распространенных причин потери зрения среди населения Земли. По данным Всемирной Организации Здравоохранения, по состоянию на 2010 г. около 126 млн. человек по всему миру больны диабетической ретинопатией (\cite{medical_zheng}). Из них 37 млн. человек имеют степень развития болезни, при которой возникает потеря зрения (\cite{medical_zheng}).

В данной работе приводится обзор основных подходов к решению задачи сегментации основных элементов глазного дна, предлагаются новые методы или модификации известных методов. В работу включены результаты вычислительных экспериментов по распознаванию элементов глазного дна, проводится сравнение предложенных методов с другими известными методами, встречающимися в литературе.

Автоматические методы сегментации элементов глазного дна могут быть использованы для построения системы диагностики болезней. Для такой системы не требуется высокая точность распознавания элементов, важно их приблизительное количество и расположение. Кроме того, компьютерная сегментация экссудатов и геморрагий может позволить обнаружить их не только врачу-профессионалу, но и, к примеру, обслуживающему персоналу лечебницы. Врач может применить компьютерную сегментацию для обнаружения элементов, не видимых невооружённым глазом, но заметных после усиления контраста или применения специальных фильтров.

\subsection{Обозначения и медицинские сведения}

Приведём описание процесса съемки глазного дна и его основных структур. Фотографии глазного дна могут быть получены с помощью так называемой фундус-камеры (см. Рис. \ref{fig:fundus_imaging}). Изображение, проецируемое на фокальную плоскость фундус-камеры, является отражением от дальней от камеры поверхности глаза. Благодаря прозрачности основных структур глаза, их можно рассмотреть на снимке глазного дна, а именно, такие элементы как оптический диск, или оптический нерв (яркое белое пятно), макулу (темное пятно), кровеносные сосуды (см. Рис. \ref{fig:normal_eye}). Оптический нерв служит средством передачи сигналов от сетчатки мозгу. Макула содержит в себе только <<колбочки>> и отвечает за высокую остроту зрения. Основные кровеносные сосуды расходятся от оптического нерва.

Некоторые глазные болезни могут быть обнаружены с помощью анализа снимков глазного дна. К примеру, глаукома характеризуется расширением оптического диска. Диабетическая ретинопатия характеризуется появлением экссудатов (белых выделений), геморрагий (черных выделений) и микроаневризм (см. Рис. \ref{fig:diabetic_eye}) (\cite{medical_dr}).

С позиций компьютерного зрения, задача диагностики глазных болезней по снимкам глазного дна состоит в построении релевантных признаков по данным изображениям и в построении модели классификации. Как показали результаты соревнования по машинному обучению Diabetic Retinopathy Detection (\cite{kaggle}), проведённого на платформе \url{kaggle.com}, удачным подходом к решению данной задачи является использование глубоких нейронных сетей. Данный подход приводит к высококачественным и робастным моделям, но не позволяет рассмотреть наличие или отсутствие отдельных признаков болезни, таких как экксудаты или геморрагии. Может быть предпочтительнее получить сегментацию этих элементов глаза в случаях, когда необходимо предоставить врачу подробный анализ снимка глазного дна пациента и автоматически выделить элементы, указывающие на болезнь.

\begin{figure}
	\centering
	\begin{subfigure}[b]{.47\textwidth}
		\includegraphics[width=\textwidth]{pics/Fundus_imaging_2.jpg}
		\caption{}
	\end{subfigure}
	~
	\begin{subfigure}[b]{.4\textwidth}
		\includegraphics[width=\textwidth]{pics/Fundus_imaging.jpg}
		\caption{}
	\end{subfigure}
	\caption{(a) Процесс съемки глазного дна при помощи фундус-камеры (источник: \url{http://tehnopuls.ru}). (b) Отражение световых лучей от поверхности глазного дна (источник: \href{http://www.opsweb.org}{Opthalmic Photographic Society}).}
	\label{fig:fundus_imaging}
\end{figure}

\begin{figure}
	\centering
	\begin{subfigure}[b]{.8\textwidth}
		\includegraphics[width=\textwidth]{pics/Eye_structure.jpg}
		\caption{}
		\label{fig:normal_eye}
	\end{subfigure}
	\hfill
	\begin{subfigure}[b]{.6\textwidth}
		\includegraphics[width=\textwidth]{pics/Fundus_structure.png}
		\caption{}
		\label{fig:diabetic_eye}
	\end{subfigure}
	\caption{(a) Основные элементы глазного яблока (источник: Kerke et al. <<Hybrid Approach for Detection of Hard Exudates>>). (b) Основные элементы глаза, поражённого диабетической ретинопатией, на изображении глазного дна (источник: \href{http://www.opsweb.org}{Opthalmic Photographic Society}). На обоих изображениях показаны: оптический диск или оптический нерв (optic nerve), макула (macula, или fovea --- её макулы), кровеносные сосуды (blood vessels). На изображении (b) показаны характерные признаки диабетической ретинопатии: экссудаты (exudates), геморрагии (hemorrhages), микроаневризмы (microaneurisms).}
	\label{fig:eye_structure}
\end{figure}

\subsection{Доступные базы данных}

В открытом доступе имеется несколько баз данных, содержащих снимки глазного дна с экспертной информацией о состоянии глаза или корректной сегментацией отдельных элементов. Ниже приведены некоторые из таких баз данных.


\begin{itemize}
	\item \textbf{\href{http://www.ces.clemson.edu/~ahoover/stare/}{STARE}} -- база данных из 400 снимков, предоставленная Shiley Eye Center, University of California, San Diego, USA и Veterans Administration Medical Center, San Diego, USA. Содержит снимки глазного дна, поражённого различными болезнями. Для каждого изображения имеется код и расшифровка диагноза. Для части изображений указана карта сосудов.
	\item \textbf{\href{http://www.adcis.net/en/Download-Third-Party/Messidor.html}{MESSIDOR}} --- база данных из 1200 снимков, предоставленных различными медицинскими учреждениями Франции. Содержит снимки глазного дна, поражённого различными болезнями. Для каждого изображения указан установленный диагноз.
	\item \textbf{\href{http://www.it.lut.fi/project/imageret/diaretdb0}{DIARETDB0}} --- база данных из 130 снимков, предоставленных Lappeenranta University of Technology, Финляндия. Из 130 снимков 20 соответствуют нормальным глазам и 110 --- поражённым диабетической ретинопатией. Для каждого снимка имеются данные о приблизительном расположении экссудатов, геморрагий, микроаневризм. 
	\item \textbf{\href{http://www.it.lut.fi/project/imageret/diaretdb1/}{DIARETDB1}}  --- база данных из 89 снимков, предоставленных Lappeenranta University of Technology, Финляндия. Из 89 снимков 5 соответствуют нормальным глазам и 84 --- поражённым диабетической ретинопатией. Для каждого снимка имеются данные о приблизительном расположении экссудатов, геморрагий, микроаневризм. Отличие от базы \textbf{DIARETDB0} в том, что в данной выборке снимки сделаны в более схожих условиях, поэтому на такой выборке легче оценивать методы, подстроенные под определённые условия освещения.
	\item \textbf{\href{https://www.kaggle.com/c/diabetic-retinopathy-detection/data}{EyePACS}} --- база данных из более чем 35000 снимков (левый и правый глаз каждого из более чем 17500 людей), предоставленных платформой EyePACS в рамках соревнования \href{https://www.kaggle.com/c/diabetic-retinopathy-detection/}{Diabetic Retinopathy Detection} (\cite{kaggle}), проведённого на платформе \url{kaggle.com}. Для каждого изображения известна степень заболевания пациента диабетической ретинопатией по шкале от 0 до 4. Особенностью выборки является большой разброс по условиям освещённости изображений.
\end{itemize}


%\subsection{Обзор литературы}

\section{Методы сегментации основных элементов глаза}

\subsection{Сегментация оптического диска}
\label{section:optic_disk}
Оптический диск представляет собой яркую область в центральной части снимка глазного дна и является изображением оптического нерва. Сегментация оптического диска важна для диагностики глазных болезней. В частности, глаукома характеризуется расширением оптического диска (\cite[стр. 13]{book_opthalmology}).

Для нахождения центра оптического диска предлагается алгоритм, описанный ниже.

\begin{enumerate}
	\item Исходное изображение переводится в цветовое пространство HSI. Пусть R, G, B --- разложения цвета пикселя по каналам R, G, B. Тогда разложения цвета пикселя по каналам H, S, I вычисляются следующим образом:
	\begin{equation}
	\centering
	\begin{split}
		\alpha &= R - \frac{1}{2} (G - B) \\ \nonumber
		\beta &= \frac{\sqrt{3}}{2} (G - B)\\
		H &= atan2(\beta, \alpha)\\
		S &= 1 - 3 \cdot \frac{\min(R, G, B)}{R + G + B}\\
		&\text{(в случае R = G = B = 0 значение } \frac{\min(R, G, B)}{R + G + B} \text{ полагается равным 0})\\
		I &= \frac{R + G + B}{3}
	\end{split}
	\end{equation}

	\item К каналу $I$ полученного изображения применяется медианная фильтрация для сглаживания изображения и удаления шумов. В результате получается изображение $I_{med}$.
	\item К изображению $I_{med}$ применяется локальная эквализация гистограммы (CLAHE, \cite[стр. 110]{szeliski}) для повышения контраста между деталями изображения. В результате получается изображение $I_{clahe}$.
	\item К изображению $I_{clahe}$ применяется отсечение по порогу. В результате получается изображение $I_{seeds}$, содержащее точки, потенциально принадлежащие оптическому диску. Также на изображении $I_{seeds}$ могут оказаться экссудаты или случайные точки, имеющие яркость, схожую с яркостью точек оптического диска.
	\item Изображение $I_{clahe}$ разбивается на связные области с помощью процедуры Region growing (см. Алг. \ref{algo:region_growing} и описание процедуры ниже). В качестве параметра seeds для процедуры Region growing используется карта точек $I_{seeds}$, таким образом, области <<разрастаются>> от точек, на месте которых в $I_{seeds}$ стоит 1. После применения данной операции получается изображение $I_{rg}$.
	\item На изображении $I_{rg}$ оставляются только те компоненты, площадь которых лежит в определённых пределах. Применяется морфологическая операция закрытия, затем операция закрытия дырок (все связные области белых пикселей закрашиваются белым цветом). Оставшееся изображение принимается в качестве итоговой карты пикселей оптического диска $I_{od}$. Радиус диска можно оценить как максимальное отклонение точек диска от его найденного центра.
\end{enumerate}

Типичный результат сегментации показан на Рис. \ref{fig:optic_disk_1}.

\begin{figure}[h]
	\begin{subfigure}{0.31\textwidth}
		\centering
		\includegraphics[width=\textwidth]{../text/pics/optic_disk/1}
		\caption{}
		\label{fig:od1}
	\end{subfigure}
	\hfill
	\begin{subfigure}{0.31\textwidth}
		\centering
		\includegraphics[width=\textwidth]{../text/pics/optic_disk/2}
		\caption{}
		\label{fig:od2}
	\end{subfigure}
	\hfill
	\begin{subfigure}{0.31\textwidth}
		\centering
		\includegraphics[width=\textwidth]{../text/pics/optic_disk/3}
		\caption{}
		\label{fig:od3}
	\end{subfigure}
	\hfill
	\begin{subfigure}{0.31\textwidth}
		\centering
		\includegraphics[width=\textwidth]{../text/pics/optic_disk/4}
		\caption{}
		\label{fig:od4}
	\end{subfigure}
	\hfill
	\begin{subfigure}{0.31\textwidth}
		\centering
		\includegraphics[width=\textwidth]{../text/pics/optic_disk/5}
		\caption{}
		\label{fig:od5}
	\end{subfigure}
	\hfill
	\begin{subfigure}{0.31\textwidth}
		\centering
		\includegraphics[width=1.05\textwidth]{../text/pics/optic_disk_1c}
		\caption{}
		\label{fig:od6}
	\end{subfigure}
	\caption{(a) Канал I исходного изображения. (b) Результат применения медианной фильтрации и CLAHE. (c) Изображение $I_{seeds}$. (d) Изображение $I_{rg}$. (e) Итоговая сегментация. (f) Результат удаления области, покрывающей диск.}
	\label{fig:optic_disk_1}
\end{figure}

Приведём описание процедуры Region growing. Процедура Region growing аналогична алгоритму BFS обхода графа в ширину, но имеет определённые модификации.  Некоторые из модификаций совпадают с предложенными в статье \cite{huiqi_li}. От каждой точки, отмеченной белым цветом в карте $I_{seeds}$, процедура строит связную компоненту приблизительно той же интенсивности, что и начальная точка. Заводится специальная очередь, изначально пустая. После того, как была выбрана точка, отмеченная белым цветом в $I_{seeds}$, рассматриваются соседние к ней точки. В конец очереди добавляются те соседние точки, для которых выполняется одно из 3 условий:
\begin{enumerate}
	\item В соседней точке нет резкого изменения цвета (можно проверить, к примеру, что значение оператора Лапласа в данной точке меньше порога $T_1$);
	\item Интенсивность соседней точки отличается от средней интенсивности построенной компоненты не более чем на $T_2$;
	\item Интенсивность соседней точки отличается от интенсивности точки, от которой началось построение компоненты, не более чем на $T_3$.
\end{enumerate}

Далее извлекается точка из начала очереди, для её незанятых соседних точек проводятся те же действия. Цикл продолжается, пока очередь не пуста. Построение компонент повторяется, пока есть нерассмотренные белые точки в $I_{seeds}$.

Многие методы используют яркость оптического диска для его нахождения. В работе \cite{huiqi_li} применяется метод главных компонент (PCA) для установления шаблонов стандартной формы оптического диска (<<собственных дисков>>). Для более точного выделения краев оптического диска могут применяться специальные методы, уточняющие его границу. В статье \cite{chernomorets} применяется преобразование Хаффа для поиска наиболее близкого круга к карте контуров в области диска и внешнее поле сил Gradient Vector Flow (\cite{GVF}) для подстройки границы. В статье \cite{huiqi_li} строится модель Active Shape Model (\cite[стр. 273]{szeliski}) и обучается на основе размеченной вручную формы диска.

\begin{algorithm}
  \caption{Процедура Region Growing}
  \begin{algorithmic}
    \Function{ region\_growing(image, radius, $T_1$, $T_2$, $T_3$[, seeds]) } {} 
    	   \State \Comment{$image$ parameter must be 1-channel image}
    	   \State m, n $\leftarrow$ image sides length
	   \State edges $\leftarrow$ pixelwise\_laplacian(image) 
	   \State q $\leftarrow$ empty queue
	   \State label $\leftarrow$ array $m$ x $n$, each element equals to -1
	   \State cluster $\leftarrow$ 0
            \For{$i$=0 to m}
            	\For{$j$=0 to n}
            		\If{label[i, j] = -1 and (seeds is not specified or seeds[i, j] = 1)}
            			\State label[i, j] $\leftarrow$ cluster
            			\State start\_point $\leftarrow$ (i, j)
            			\State clear queue q, then append start\_point element to it
            			\While{queue q is not empty}
            				\State ($last_i$, $last_j$) $\leftarrow$ element popped from the left of q
            				\For{each element ($n_i$, $n_j$) in square neighborhood with center ($last_i$, $last_j$) and radius $radius$}
            					\If{label[$n_i$, $n_j$] = -1 and borders[i, j] < $T_1$ and \\
            					|image[$n_i$, $n_j$] - mean intensity of growing cluster| < $T_2$ and \\
            					|image[$n_i$, $n_j$] - image[start\_point]| < $T_3$}
            						\State label[$n_i$, $n_j$] $\leftarrow$ cluster
            						\State append ($n_i$, $n_j$) to q
            					\EndIf
            				\EndFor
            			\EndWhile
            			\State cluster $\leftarrow$ cluster + 1
            		\EndIf
            	\EndFor
            \EndFor
            
            return label
    \EndFunction
  \end{algorithmic}
  \label{algo:region_growing}
\end{algorithm}

Автором были получены следующие результаты: на данных выборки \linebreak \textbf{DIARETDB1}, состоящей из 89 снимков, метод правильно оценил центр и радиус оптического диска (с точностью приблизительно в 10 пикселей) в 81\% случаев и правильно локализовал оптический диск в прямоугольной окрестности в 83\% случаев. В работе \cite{jose} предложен упрощённый по сравнению с описанным выше метод распознавания: провести бинаризацию канала G исходного изображения, разбить на связные компоненты и выбрать из них максимальную по площади. Данный метод показывает низкие результаты на выборке \textbf{DIARETDB1}. Стоит отметить, что в такой метод необходимо добавить предварительное усиление контраста (эквализацию гистограммы или CLAHE), иначе нельзя выбрать общий порог бинаризации для различных снимков. Кроме того, даже после предобработки максимальная связная компонента не всегда соответствует оптическому диску. Предлагаемый метод отработал лучше на данной выборке, чем метод из работы \cite{huiqi_li}, в основном потому, что метод из работы \cite{huiqi_li} не учитывает размер областей-кандидатов на оптический диск. Ошибки упомянутого метода приводят к выбору ярких областей, отличных от оптического диска, поэтому если данный метод сработал некорректно, то центр диска будет оценён с большой ошибкой. 

\subsection{Сегментация кровеносных сосудов}
\label{vessels_segmentation}

Сегментация кровеносных сосудов необходима как для диагностики глазных болезней (\cite{medical_dr}), так и как вспомогательная процедура для выделения других элементов. Кровеносные сосуды можно сегментировать, используя их цветовые и геометрические характеристики. На каждом участке сосуда, не содержащем разветвлений, сосуд можно приблизить некоторой ломаной из небольшого числа звеньев. Поэтому можно получать сегментацию сосудов, применяя фильтры, распознающие наличие линий определённого направления и толщины на изображении. При построении алгоритма сегментации кровеносных сосудов необходимо учитывать сложности, связанные с вариацией толщины сосуда и локального контраста вдоль длины сосуда. Популярным подходом к обнаружению кровеносных сосудов является применение фильтров Габора. Фильтр Габора выделяет определённые линии на изображении, повышает их контраст по сравнению с остальной частью изображения. Желаемое направление и толщина линий определяется параметрами ядра фильтра. Классический фильтр Габора имеет следующее ядро (\cite{qui_li}):

\begin{equation}
\begin{split}
&g_{\sigma_x, \sigma_y, f, \theta}(x, y) = \exp\left[ -\pi \left(  \frac{x'^2}{\sigma_x^2} + \frac{y'^2}{\sigma_y^2} \right) \right] \exp\left( 2\pi i f x' \right) \\
&x' = x\cos\theta + y\sin\theta \\ 
&y' = -x\sin\theta + y\cos\theta \\
&x = \overline{-r, r}, \, y = \overline{-r, r},
\end{split}
\label{formula:gabor}
\end{equation}

где $r$ --- радиус ядра, $\sigma_x, \sigma_y$ --- стандартные отклонения Гауссианы фильтра, $f$ --- частота волновой составляющей фильтра, $\theta$ --- направление детектируемых линий (угол в радианах).

На практике применяется свертка с вещественной частью фильтра Габора. В работе \cite{qui_li} предлагается использовать параметры, подобранные специальным образом для задачи выделения кровеносных сосудов: 
\begin{equation}
\begin{split}
&\sigma_x = s, \, \text{   параметр s определяется отдельно}, \\
&\sigma_y = \kappa s,\\
&f = \frac{n\lambda}{\alpha \pi s}, \\ 
&\text{n --- число направлений, по которым вычисляется отклик фильтра}, \\
&\theta = 2\pi \frac{i}{n},\, i = \overline{0, n-1} \\
&\alpha \in [1, 1.5],\, \beta \in [0.5, 1], \kappa = 0.85, \lambda = \sqrt{\frac{2 \ln{2}}{\pi}} \\
\end{split}
\label{formula:gabor_params}
\end{equation}

Варьируя параметры ядра фильтра, можно получать изображения, в котором выделены линии разной толщины и разных направлений. В данной задаче можно взять максимум из откликов фильтров по разным значениям параметра $\theta$ в каждом пикселе.

Автором было обнаружено, что свёртка предобработанных изображений с ядром классического фильтра Габора (\ref{formula:gabor}) даёт сильно зашумлённое изображение. Поэтому целесообразно применить к изображению операции предобработки и предварительного усиления контраста. Ниже приводится полное описание метода.

\begin{enumerate}
\item Исходное изображение RGB переводится в цветовое пространство HSI (см. раздел \ref{section:optic_disk}), далее производится работа с каналом I.
\item К каналу $I$ применяется локальная эквализация гистограммы (CLAHE) для повышения контраста между деталями изображения. В результате получается изображение $I_{clahe}$.
\item К изображение $I$ применяется медианная фильтрация. В результате получается изображение $I_{med}$. Изображение $I_{med}$ вычитается из $I_{clahe}$ для удаления цветовой информации о фоне: $I_{sc} = I_{clahe} - I_{med}$.
\item На изображении $I_{sc}$, как и на исходном изображении, сосуды представляют собой тёмные линии, а фон в целом более светлый, чем сосуды. Чтобы сосуды соответствовали высоким значениям интенсивности, а остальной фон более низким, изображение $I_{sc}$ инвертируется: $I_{inv} = 255 - I_{sc}$.
\item Производится свертка изображения $I_{inv}$ с каждым фильтром из банка фильтров Габора, соответствующих различным параметрам $\theta$. Ядро каждого фильтра определяется формулой (\ref{formula:gabor}), параметры определяются формулами (\ref{formula:gabor_params}), $s = 16$ для выборки \textbf{DIARETDB1} или $s = 12$ для выборки \textbf{STARE}). В каждой точке выбирается максимум отклика по всем фильтрам, соответствующим различным $\theta$. В результате получается изображение $I_{vessels}$.
\item Проводится бинаризация изображения $I_{vessels}$. Порог бинаризации выбирается как значение интенсивности $t$, такое что $N\%$ пикселей изображения $I_{vessels}$ темнее $t$. Автором было выбрано $N\% = 95\%$. После бинаризации с изображения удаляются связные компоненты, площадь которых ниже заданного порога. В результате получается требуемая карта сосудов.
\end{enumerate}

Типичный результат сегментации сосудов приведён на Рис. \ref{fig:vessels_1}.

\begin{figure}[t]
	\centering
	\begin{subfigure}[b]{.31\textwidth}
		\includegraphics[width=1.01\textwidth]{pics/vessels_1a.png}
		\caption{}
		\label{fig:vessels_1a}
	\end{subfigure}
	\hfill
	\begin{subfigure}[b]{.31\textwidth}
		%\centering
		\includegraphics[width=\textwidth]{../presentation/pics/vessels/2_1}
		\caption{}
		\label{fig:2_1}
	\end{subfigure}
	\hfill
	\begin{subfigure}[b]{.31\textwidth}
		\includegraphics[width=\textwidth]{../presentation/pics/vessels/3_1}
		\caption{}
		\label{fig:vessels3_1}
	\end{subfigure}
	
	\begin{subfigure}[b]{.31\textwidth}
		\centering
		\includegraphics[width=\textwidth]{../presentation/pics/vessels/4_1}
		\caption{}
		\label{fig:vessels4_1}
	\end{subfigure}
	\hfill
	\begin{subfigure}[b]{.31\textwidth}
		\includegraphics[width=\textwidth]{pics/vessels_1b.png}
		\caption{}
		\label{fig:vessels_1b}
	\end{subfigure}
	\hfill
	\begin{subfigure}[b]{.31\textwidth}
		\includegraphics[width=\textwidth]{pics/vessels_1c.png}
		\caption{}
		\label{fig:vessels_1c}
	\end{subfigure}
	
	
	\caption{(a) Канал I исходного изображения. (b) Изображение $I_{sc}$. (c) Изображение $I_{inv}$. (d) Изображение $I_{vessels}$. (e) Результат сегментации. (f) Корректная разметка из базы \textbf{STARE}.}
	\label{fig:vessels_1}
\end{figure}

Автором были получены следующие результаты: на части выборки \textbf{STARE} из 20 снимков, для которой доступна экспертная разметка кровеносных сосудов, метод достиг чувствительности (доли правильно классифицированных пикселей среди не принадлежащих сосудам) 90\% и специфичности (доли правильно классифицированных пикселей среди принадлежащих сосудам) 76\%. Недостатком метода является высокая чувствительность к шумам, а также к экссудатам и геморрагиям. Экссудаты и геморрагии нередко ошибочно распознаются алгоритмом как части сосудов. Для устранения этой проблемы целесообразно удалить экссудаты и геморрагии по описанным в следующих разделах процедурам (для геморрагий можно использовать процедуру без удаления сосудов --- см. раздел~\ref{subsection:hemorrhages}). На выборке \textbf{STARE}, тем не менее, такой способ не привёл к значительному повышению качества.

Аналогичная техника выделения сосудов с помощью фильтра Габора была применена в работе \cite{chernomorets}. Для уточнения карты сосудов был применён метод морфологических амёб, описанный в статье. Метод заключается в дополнении бинарной карты сосудов теми пикселями, которые близки к уже размеченным и по расстоянию, и по цвету. Автором было обнаружено, что данный метод позволяет экстраполировать сосуды на те участки, которые не были обнаружены, однако слишком сильно увеличивает области, ошибочно сегментированные как экссудаты или геморрагии. Тем самым, данный метод не устраняет данную ошибку и понижает специфичность распознавания. В литературе встречаются и принципиально другие методы, --- к примеру, в работе \cite{choras} сосуды извлекаются на основе детектора краёв Canny. Данный подход также обладает низкой специфичностью, т.~к. извлекает произвольные области с резкими краями.

\subsection{Сегментация экссудатов} 
\label{section:exudates}

Экссудаты представляются на снимке глазного дна яркими белыми пятнами произвольной, как правило изрезанной формы. Для сегментации экссудатов предлагается метод, основанный на материале, изложенном в \cite{karegowda}. Ниже приводится описание предлагаемого алгоритма. Метод основан на построении модели на основе стратегии обучения с учителем на уровне пикселей. Экспертную сегментацию можно получить, например, из базы данных \textbf{DIARETDB1}.

\begin{enumerate}
	\item Исходное изображение переводится в цветовое пространство HSI (см. раздел \ref{section:optic_disk}).
	\item К каналу I полученного изображения применяется медианная фильтрация для сглаживания изображения и удаления шумов. В результате получается изображение $I_{med}$.
	\item К изображению $I_{med}$ применяется локальная эквализация гистограммы (CLAHE, \cite[стр. 110]{szeliski}) для повышения контраста между деталями изображения. В результате получается изображение $I_{clahe}$.
	\item Экссудаты и оптический диск часто имеют схожие значения яркости, поэтому для сегментации экссудатов целесообразно удалить оптический диск с изображения. Для этого применяется описанная процедура сегментации оптического диска (см. раздел \ref{section:optic_disk}). Оптический диск локализуется в рамках прямоугольной окрестности, данная окрестность далее не рассматривается при анализе снимка.
	\item Осуществляется перебор точек изображения $I_{clahe}$ с некоторым шагом. Для каждой точки, принадлежащей рассматриваемой области, записываются в таблицу её признаки:
	\begin{itemize}
		\item \textbf{hue} --- значение канала H в данной точке
		\item \textbf{intensity} --- значение канала I в данной точке
		\item \textbf{mean intensity} --- среднее значение канала I в окрестности данной точки
		\item \textbf{std intensity} --- стандартное отклонение значений канала I в окрестности данной точки
		\item \textbf{distance to optic disk} --- расстояние до центра оптического диска
	\end{itemize}
	Для данных пикселей также собирается в таблицу информация о принадлежности пикселей областям, занятым экссудатами.
	\item Описанная выше процедура осуществляется для каждого изображения выборки. Далее выбирается классификатор, который восстанавливает закономерность между признаками пикселей и принадлежностью областям, занятым экссудатами. Автором был выбран классификатор Random Forest как выдающий результат, обладающий наибольшим средним гармоническим между специфичностью (долей правильно угаданных пикселей среди принадлежащих экссудатам) и чувствительностью (долей правильно угаданных пикселей среди не принадлежащих экссудатам) среди рассмотренных моделей на выборке \textbf{DIARETDB1}.
	
\end{enumerate}

\begin{figure}[h]
	\begin{subfigure}[b]{.45\textwidth}
		\includegraphics[width=\textwidth]{pics/exudates/1_1}
		\caption{}
		\label{fig:exudates1_1}
	\end{subfigure}
	\hfill
	\begin{subfigure}[b]{.45\textwidth}
		\includegraphics[width=\textwidth]{pics/exudates/2_1}
		\caption{}
		\label{fig:exudates_1a}
	\end{subfigure}
	
	\begin{subfigure}[b]{.45\textwidth}
		\includegraphics[width=1\textwidth]{pics/exudates_1b_1.jpg}
		\caption{}
		\label{fig:exudates_1b}
	\end{subfigure}
	\hfill
	\begin{subfigure}[b]{.45\textwidth}
		\includegraphics[width=1\textwidth]{pics/exudates_1c_1.jpg}
		\caption{}
		\label{fig:exudates_1c}
	\end{subfigure}
	\caption{(a) Канал I исходного изображения. (b) Изображение $I_{clahe}$. (c) Результат сегментации. (d) <<Корректная>> сегментация из базы \textbf{DIARETDB1}.}
	\label{fig:exudates_1}
\end{figure}

Типичный результат сегментации показан на Рис. \ref{fig:exudates_1}. В базе данных \textbf{DIARETDB1} для каждого снимка приведена корректная экспертная сегментация экссудатов и геморрагий, однако области, обозначенные на изображениях с корректной сегментацией, нельзя использовать в качестве правильного ответа, т.~к. они являются объединениями областей, полученных от разных врачей. Поэтому специфичность и чувствительность исследуемого метода не интерпретируемы на данной выборке. Авторами статьи \cite{karegowda} с помощью того же алгоритма и нейронной сети в качестве алгоритма классификации был получен следующий результат на использованной ими выборке: чувствительность --- 96.97\%, специфичность --- 100\%. 

Другие методы выделения экссудатов предполагают иные процедуры предобработки и обучения. В работе \cite{huiqi_li} экссудаты извлекаются в цветовом пространстве LUV с помощью процедуры, похожей на Region growing (см. Алг.~\ref{algo:region_growing}), с применением детектора краёв Canny. Было установлено, что данный метод способен воспроизвести достаточно точную сегментацию поражений глаза (и экссудатов, и геморрагий), но сильно зависит от порога, и может выдавать очень разные результаты для разных снимков. Таким образом, данный метод не удалось применить к большой выборке изображений. 
Некоторые авторы не используют обучение с учителем для поиска экссудатов. В работе \cite{chernomorets} они извлекаются как пиксели, в окрестности которых дисперсия интенсивностей выше заданного порога. Этот метод настроен на поиск экссудатов в области макулы, поскольку именно экссудаты, близкие к макуле, вызывают потерю зрения (\cite{medical_dr}). Описанный выше метод позволяет найти экссудаты на всём изображении, что может быть предпочтительнее.

%В работе \cite{zhang_chutape} применяется техника кластеризации Fuzzy C-Means для поиска выделяющихся фрагментов на изображении в цветовом пространстве LUV, затем применяется классификатор SVM для отбрасывания неподходящих областей. 
% (это предложение должно быть не в конце, а перед ``Некоторые авторы не используют обучение с учителем для поиска экссудатов'')

\subsection{Сегментация геморрагий}
\label{subsection:hemorrhages}

Геморрагии представляют собой тёмные пятна и, так же как и экссудаты, указывают на наличие диабетической ретинопатии. Сегментация геморрагий является достаточно сложной задачей, т.~к. их средний цвет практически совпадает с цветом фона, и они, как правило, не так контрастны относительно фона, как экссудаты или оптический диск. 

Удачным подходом к повышению контраста геморрагий является применение к изображению специального оператора, в основе которого лежит применение гауссовского фильтра высоких частот (впервые применён для сегментации геморрагий в работе \cite{sinthanayothin} под названием <<Moat Operator>>). Идея фильтра состоит в том, чтобы убрать из изображения <<медленные>> изменения цвета. Приведём описание Moat Operator на основе работы \cite{jose}.

\begin{enumerate} 
\item Исходное изображение $g(x, y)$ размера $N$ x $M$ преобразуется из пространства координат в пространство частот с помощью двумерного преобразования Фурье:
\begin{equation}
\begin{split}
	&G(u, v) = \frac{1}{N} \sum\limits_{x=0}^{N - 1} \sum\limits_{y=0}^{M - 1} g(x, y) \exp\left[  -2\pi i \left( \frac{ux + vy}{N} \right) \right], \\ \nonumber
	&u =\overline{0, N - 1},\, v = \overline{0, M - 1}
\end{split}
\end{equation}

\item К изображению $G(u, v)$ применяется гауссовский фильтр высоких частот с параметром $\sigma$:
\begin{equation}
\begin{split}
	&H(u, v) = 1 - \exp\left( -\frac{u^2 + v^2}{2\sigma^2} \right), \\ \nonumber
	&I(u, v) = H(u, v) \cdot G(u, v), \\
	&u =\overline{0, N - 1}, \,v = \overline{0, M - 1}
\end{split}
\end{equation}

В результате получается изображение $I(u, v)$ в частотном пространстве.
\item Изображение $I(u, v)$ переводится из частотного пространства в координатное пространство:
\begin{equation}
\begin{split}
	&i(x, y) = \sum\limits_{u=0}^{N - 1} \sum\limits_{v=0}^{M - 1} I(u, v) \exp\left[  2\pi i \left( \frac{ux + vy}{N} \right) \right], \\ \nonumber
	&x =\overline{0, N - 1},\, y = \overline{0, M - 1}
\end{split}
\end{equation}

\item Moat Operator от исходного изображения $g(x, y)$ определяется следующим образом:
\begin{equation}
\begin{split}
	&g_{moat}(x, y) = g(x, y) - |i(x, y)|, \\ \nonumber
	&x =\overline{0, N - 1},\, y = \overline{0, M - 1}
\end{split}
\end{equation}
\end{enumerate}

Вместо подсчета по стандартным формулам двумерного прямого и обратного преобразования Фурье могут использоваться алгоритмы Быстрого преобразования Фурье (\cite[стр. 134]{szeliski}), сложность которых для изображения $N$ x $M$ составляет $\mathcal{O}(NM \log(MN))$ арифметических операций (в отличие от $\mathcal{O}\left((NM)^2\right)$ в случае вычисления по приведённым формулам). 

Пример применения Moat Operator (с $\sigma = 5$) представлен на Рис.~\ref{fig:hem1} (b). Можно наблюдать, что после применения оператора на изображении остаются в основном сосуды и геморрагии, а остальные элементы снимка становятся незаметны. В результате применения оператора получается одноканальное изображение в оттенках серого. К полученному изображению необходимо применить процедуру постобработки. Ниже приведено полное описание предлагаемого алгоритма сегментации.

\begin{enumerate}
\item К каналу G исходного изображения в цветовом пространстве RGB применяется Moat Operator. В результате получается изображение $G_{moat}$.
\item К изображению $G_{moat}$ применяется локальная эквализация гистограммы (CLAHE, \cite[стр. 110]{szeliski}) для повышения контраста между деталями изображения. В результате получается изображение $G_{clahe}$.
\item К изображению $G_{clahe}$ применяется медианная фильтрация для сглаживания изображения и удаления шумов. В результате получается изображение $G_{med}$.
\item Производится бинаризация изображение $G_{med}$. Порог бинаризации вычисляется по методу Otsu (\cite{otsu}), который состоит в поиске порога, минимизирующего дисперсию интесивности между передним планом и фоном. Итоговый порог выбирается как 90\% от порога, подсчитанного по методу Otsu. В результате получается изображение $G_{thresholded}$.
\item Производится выделение кровеносных сосудов на исходном изображении с помощью процедуры, описанной в разделе \ref{vessels_segmentation}. Над полученной картой сосудов производится морфологическая операция дилатации с квадратным ядром размера $K$ x $K$ (автором был выбран размер $K = 10$ для базы \textbf{DIARETDB1}). Тем самым сегментированные сосуды утолщаются. В позициях, где в полученной карте сосудов отмечено их присутствие, на изображении $G_{thresholded}$ выставляются черные пиксели. Таким образом, сосуды удаляются из кандидатов на геморрагии.
\item Изображение $G_{thresholded}$ разбивается на связные компоненты. Из всех компонент оставляются компоненты, площадь и эксцентриситет которых находятся в определённом диапазоне. Эксцентриситет области равен отношению сторон минимального произвольно ориентированного прямоугольника, содержащего данную область (\cite[стр. 6]{shapes}). В результате получается требуемая карта геморрагий.

\end{enumerate}

Типичный результат сегментации геморрагий показан на Рис.~\ref{fig:hem1}. Предложенный метод основан на материале, предложенном в работе \cite{jose}. Основное улучшение описанного метода состоит в подсчете порога бинаризации на основе метода Otsu. Было обнаружено, что при использовании фиксированного порога бинаризации результат сегментации сильно зависит от цветовых свойств изображения.

Стоит заметить, что шаг 5 сегментации --- удаление кровеносных сосудов --- не является обязательным, т.~к. сосуды как связные компоненты имеют высокий эксцентриситет, и подавляющее большинство из них может быть удалено на шаге 6. Целесообразно отказаться от шага 5 в том случае, если процедура удаления сосудов чувствительна к геморрагиям.

\begin{figure}[t]
	\centering
	\begin{subfigure}[b]{.31\textwidth}
		\includegraphics[width=\textwidth]{pics/hemorrhages/green_1.png}
		\caption{}
		\label{fig:moat_a}
	\end{subfigure}
	\hfill
	\begin{subfigure}[b]{.31\textwidth}
		\includegraphics[width=\textwidth]{pics/hemorrhages/1_1.png}
		\caption{}
		\label{fig:moat_b}
	\end{subfigure}
	\hfill
	\begin{subfigure}[b]{.31\textwidth}
		\centering
		\includegraphics[width=1.0\linewidth]{pics/hemorrhages/2_1}
		\caption{}
		\label{fig:hem2_1}
	\end{subfigure}
	
	\begin{subfigure}[b]{.31\textwidth}
		\centering
		\includegraphics[width=1.0\linewidth]{pics/hemorrhages/3_1}
		\caption{}
		\label{fig:hem3_1}
	\end{subfigure}
	\hfill
	\begin{subfigure}[b]{.31\textwidth}
		\centering
		\includegraphics[width=1.0\linewidth]{pics/hemorrhages/4_1}
		\caption{}
		\label{fig:hem4_1}
	\end{subfigure}
	\hfill
	\begin{subfigure}[b]{.31\textwidth}
		\centering
		\includegraphics[width=1.0\linewidth]{pics/hemorrhages/final_1}
		\caption{}
		\label{fig:hem_final_1}
	\end{subfigure}
	
	\caption{(a) Канал G исходного изображения. 
		(b) Изображение $G_{moat}$. 
		(c) Изображение $G_{med}$.
		(d) Изображение $G_{thresholded}$.
		(e) Результат удаления сосудов с изображения $G_{thresholded}$.
		(f) Результат сегментации.}
	\label{fig:hem1}
\end{figure}

%В литературе встречаются и другие подходы к выделению геморрагий. В работе \cite{sinthanayothin} области извлекаются с помощью алгоритма типа region growing. Некоторые подходы основаны на обучении на основе экспертной разметки. В работе \cite{chernomorets} геморрагии извлекаются с помощью медианной фильтрации. Затем полученная карта геморрагий разбивается на области, для каждой области подсчитываются геометрические и цветовые признаки, и используется классификатор SVM для обучения на уровне областей по экспертным данным. Похожий подход использовался в работе \cite{zhang_chutape}: области классифицируются при помощи обучения нескольких моделей SVM, каждая из которых соответствуют изображению с изменённым масштабом. Для каждой области вероятность принадлежать геморрагиям определяется как максимальная вероятность среди оценённых всеми моделями. 

В работе \cite{sinthanayothin} при использовании Moat Operator и сегментации с помощью алгоритма типа Region growing (см. Алг. \ref{algo:region_growing}) были получены следующие результаты: на использованной авторами выборке из 14 снимков чувствительность составила 77.5\%, специфичность --- 88.7\%.

Автором были исследованы и другие методы сегментации геморрагий. Было выяснено, что подходы, основанные на обучении с учителем, аналогичные описанному в разделе~\ref{section:exudates}, не приводят к положительным результатам для геморрагий при использовании выборки \textbf{DIARETDB1}. Возможно, причиной этого явилось отсутствие у автора корректной экспертной разметки геморрагий. В работе \cite{chernomorets} геморрагии извлекаются с помощью медианной фильтрации, затем полученная карта геморрагий разбивается на области, для каждой области подсчитываются геометрические и цветовые признаки, и используется классификатор SVM для обучения на уровне областей по экспертным данным. Автор данной работы реализовал данный метод с использованием следующих признаков областей: площадь, округлость (отношение площади к квадрату периметра), эксцентриситет, стандартное отклонение интенсивности в области, средний цвет области в канале R и G. Положительных результатов не было получено. Метод, изложенный в работе \cite{sinthanayothin}, заключающийся в использовании Moat Operator и сегментации с помощью алгоритма типа Region growing, привёл к неплохим результатам, но оптимальные параметры алгоритма Region growing в данном случае сильно зависят от цветовых свойств изображения. 

Описанный метод настроен на поиск крупных геморрагий, однако может быть целесообразно выделение и мелких (<<точечных>>) геморрагий, к примеру, для диагностики болезни на ранней стадии. В некоторых работах предложены методы поиска точечных геморрагий. К примеру, в работе \cite{niemeijer} предложен метод, основанный на усилении контраста с помощью гауссовского фильтра и построении модели классификации KNN с более чем 20 признаками яркости и формы. Метод достигает 100\% чувствительности и 87\% специфичности на использованной авторами статьи выборке из 100 изображений.

\section{Заключение}

Компьютерное исследование глазных снимков --- это достаточно перспективное направление исследований, которое может существенно повысить уровень медицинской диагностики глазных болезней. В работе приведены улучшенные версии известных методов сегментации основных элементов глазного дна с подробным описанием. Показано, что описанные структуры снимка глазного дна поддаются сегментации с достаточно высокой точностью. Описанные алгоритмы можно в дальнейшем улучшать, подстраивать под определённые условия освещения или свойства аппарата, осуществляющего процесс съёмки.

При реализации описанных методов использовался язык программирования Python 2.7 с библиотеками: scikit-learn 0.17.1 (содержит реализации алгоритмов машинного обучения), scikit-image 0.12.3 (содержит реализации некоторых алгоритмов обработки изображений), mahotas 1.4.1 (содержит эффективные реализации некоторых фильтров и морфологических операций).
\nocite{*}

\newpage

\renewcommand{\bibname}{Список литературы}
\addcontentsline{toc}{section}{\bibname}

%\begin{thebibliography}{9}
%\bibliography{references}
%bibliographystyle{plain}
%\bibliography{references}{}
%\bibliographystyle{plain}
%\end{thebibliography}

\printbibliography


\end{document}